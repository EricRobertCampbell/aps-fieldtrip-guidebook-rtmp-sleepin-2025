\documentclass[11pt,letterpaper]{article}

% Packages
\usepackage[utf8]{inputenc}
\usepackage[T1]{fontenc}
\usepackage[margin=1in]{geometry}
\usepackage{graphicx}
\usepackage{hyperref}
\usepackage[style=authoryear,backend=biber]{biblatex}
\usepackage{booktabs}
\usepackage{float}
\usepackage{parskip}
\usepackage{pdfpages}

% Bibliography resource
\addbibresource{bibliography.bib}

% Hyperref setup
\hypersetup{
    colorlinks=true,
    linkcolor=blue,
    filecolor=magenta,
    urlcolor=cyan,
    citecolor=blue,
    pdftitle={APS Field Trip Guidebook - Royal Tyrrell Museum},
    pdfauthor={Alberta Palaeontological Society},
}

% Title information
\title{Alberta Palaeontological Society Field Trip\\
Royal Tyrrell Museum of Palaeontology\\
Drumheller, Alberta}
\author{Alberta Palaeontological Society}
\date{November 2025}

\begin{document}

\maketitle

\tableofcontents
\newpage

\section{Introduction}

Welcome to the Alberta Palaeontological Society field trip to the Royal Tyrrell Museum of Palaeontology in Drumheller, Alberta.

\subsection{Schedule}

\begin{center}
\textbf{Day 1}
\vspace{0.5em}

\begin{tabular}{@{}p{0.25\textwidth}p{0.65\textwidth}@{}}
\toprule
\textbf{Time} & \textbf{Activity} \\
\midrule
6:30 p.m.--7:00 p.m. & Welcome / Orientation \\
\midrule
Evening & Educational Programming Sessions \\
\midrule
Late Evening & Bedtime Snack in Museum Cafeteria \\
\midrule
Night & Bedtime! Make up campsite in Dinosaur Hall \\
\bottomrule
\end{tabular}
\end{center}

\vspace{1em}

\begin{center}
\textbf{Day 2}
\vspace{0.5em}

\begin{tabular}{@{}p{0.25\textwidth}p{0.65\textwidth}@{}}
\toprule
\textbf{Time} & \textbf{Activity} \\
\midrule
7:30 a.m. & Rise and Shine \\
\midrule
Morning & Breakfast in Museum Cafeteria \\
\midrule
Morning & Video in Museum Auditorium \\
\midrule
Morning & Museum Shop open for souvenirs \\
\midrule
10:00 a.m.--5:00 p.m. & Museum Opens! Free admission to galleries \\
\bottomrule
\end{tabular}
\end{center}

\subsection{Arrival \& Departure + Museum Map}

\paragraph{Welcome} Arrive at the Museum at 6:30 p.m. to 7:00 p.m.
\begin{itemize}
    \item Camp-In staff will unlock the Tyrrell Learning Centre doors at 6:30 p.m. to allow access to the facility.
    \item The Museum closes at 5:00 p.m. and is locked and unavailable until 6:30 p.m.
\end{itemize}

\paragraph{Drop-Off} Go straight on the one-way road and around the hill to the drop-off zones on the attached diagram.
\begin{itemize}
    \item Please unload your gear in the ``PUBLIC DROP-OFF AND PICK-UP ZONE'' just before the fire lane.
    \item Idling or parking in the Fire Lane is NOT permitted at any time.
    \item After unloading, the vehicle(s) should be parked in the general parking lot.
    \item As your group exits the vehicles, the GROUP ORGANIZER should enter the Tyrrell Learning Centre lobby (marked with ``CAMP-IN ENTRANCE'' on the map).
\end{itemize}

\includepdf[pages=2,pagecommand={}]{4 - arrival departure 2025-26.pdf}

\paragraph{Get Settled In}
\begin{itemize}
    \item The Host/Camp-In staff will greet you in the lobby and provide you with your registration package.
    \item Please note your assigned group number(s).
    \item CAMPERS should go to the Tyrrell Learning Centre entrance with their gear.
    \item Everyone should wait at their assigned group number (posted in the lobby) until the entire group has arrived.
    \item Camp-In staff will escort you into Dinosaur Hall; listen carefully to the instructions about sleeping arrangements. Everyone may drop off their gear, but there is no need to unpack as there will be plenty of time to do so at bedtime.
    \item Go directly back to the Tyrrell Learning Centre lobby and proceed into the Auditorium for orientation.
\end{itemize}

\paragraph{Let the Fun Begin}
After the orientation, the Host will lead you back to the Tyrrell Learning Centre for exciting palaeo programming until bedtime snack.

\subsubsection*{Departure Procedures: Saturday/Sunday Morning}

\paragraph{Rise and Shine} Wake up call is at 7:30 a.m.
\begin{itemize}
    \item Pack up your gear, bring it to the lobby, and have it ready to load into your vehicle(s) in the drop-off/pick-up zones from 7:30--8:45 a.m.
    \item Breakfast is served in the Cafeteria beginning at 8:15 a.m. Due to limited storage space, all gear must be out of the building before 9:00 a.m.
\end{itemize}

\paragraph{Gallery Visit}
\begin{itemize}
    \item The Museum Shop opens at 9:30 a.m.
    \item Beginning at 10:00 a.m. you have FREE ADMISSION to the Museum for the day. Enjoy your visit!
\end{itemize}

\subsection{Packing List}

When packing, bring all of the items you would need normally for a sleepover. Some ideas are:

\begin{itemize}
    \item Comfortable sleeping clothes/pajamas
    \item Facecloth, small hand towel, toothbrush, and toothpaste
    \item Sleeping bag, twin or double sized foam/air mattress and pillow
    \item Money for Museum Shop purchases
    \item Earplugs
    \item A small flashlight
    \item Water bottle
\end{itemize}

\subsection{Risks \& Hazards}

While most activities during this field trip will take place inside the Royal Tyrrell Museum under the supervision of trained museum staff, participants should be aware that hazards still exist. This event takes place in November when winter weather conditions are common in Alberta. Participants should exercise caution when driving to and from the museum, as roads may be icy, snow-covered, or subject to reduced visibility. Please check weather and road conditions before traveling and allow extra time for your journey.

\subsection{Disclaimer}

While every effort has been made to verify the accuracy of the information presented in this guide users are referred to the original sources listed in the reference section. Access conditions may change without notice. 

Outdoor activities may expose you to dangers. These include, but are not limited to getting lost, encountering bears, aggressive wildlife, insect bites and diseases, rough trails, slippery footing, inclement weather, lightning, forest fires, falling trees, contaminated drinking water, isolation from help, difficult evacuation, accidents and other risks and hazards both of a natural and man-made origin. 

You are responsible for your well-being in the backcountry. Neither the author of this guide nor the Alberta Palaeontological Society, its affiliates and all respected members, officers, directors, employees, agents and contractors can be held responsible for any difficulties, personal injury, income or property loss, illness or death that arises from using the information in this guide.

\section{Background Information}
\subsection{Royal Tyrrell Museum of Palaeontology}
\subsection{Drumheller and Surrounding Area}
\section{References}

\printbibliography[heading=none]

\end{document}
